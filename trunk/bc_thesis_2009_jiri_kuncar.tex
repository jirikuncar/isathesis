%% LyX 1.6.2 created this file.  For more info, see http://www.lyx.org/.
%% Do not edit unless you really know what you are doing.
\documentclass[12pt,czech,notitlepage]{report}
\usepackage[T1]{fontenc}
\usepackage[utf8]{inputenc}
\pagestyle{plain}
\setcounter{secnumdepth}{3}
\setcounter{tocdepth}{3}
\usepackage{verbatim}
\usepackage{graphicx}

%%%%%%%%%%%%%%%%%%%%%%%%%%%%%% User specified LaTeX commands.
\usepackage{a4wide}
\usepackage[left=4cm]{geometry}

\usepackage{babel}

\begin{document}
\begin{titlepage}\vspace{15mm}


\begin{center}
{\large Univerzita Karlova v Praze}\\
{\large Matematicko-fyzikální fakulta\vspace{5mm}
}
\par\end{center}{\large \par}

\begin{center}
\textbf{\Large BAKALÁŘSKÁ PRÁCE\vspace{10mm}
}
\par\end{center}{\Large \par}

\begin{center}
\includegraphics[scale=0.3]{logo}\vspace{15mm}

\par\end{center}

\begin{center}
{\Large Jiří Kunčar \vspace{5mm}
}
\par\end{center}{\Large \par}

\begin{center}
\textbf{\Large Informační systém pro jazykovou agenturu\vspace{5mm}
}
\par\end{center}{\Large \par}

\begin{center}
\textbf{\large Ústav formální a aplikované lingvistiky\vspace{15mm}
}
\par\end{center}{\large \par}

\noindent \begin{center}
{\large Vedoucí bakalářské práce: RNDr. Miroslav Spousta \vspace{1mm}
}
\par\end{center}{\large \par}

\noindent \begin{center}
{\large Studijní program: informatika, správa počítačových systémů\vspace{20mm}
}
\par\end{center}{\large \par}

\begin{center}
{\large 2009}
\par\end{center}{\large \par}

\end{titlepage}\setcounter{page}{2}

\vspace{10mm}


\noindent Děkuji panu RNDr. Miroslavu Spoustovi za pomoc, připomínky,
cenné rady a za odborné vedení bakalářské práce. Dále bych rád poděkoval
firmě Primalingua s.r.o. za poskytnutí prostředků na vývoj aplikace.
Speciální poděkování patří především paní majitelce Mgr. Miluši Psotové
a paní RNDr. Jitce Kunčarové, která vývoj informačního systému iniciovala.

\vfill{}


\noindent Prohlašuji, že jsem svou bakalářskou práci napsal(a) samostatně
a výhradně s použitím citovaných pramenů. Souhlasím se zapůjčováním
práce a jejím zveřejňováním.\bigskip{}


\noindent V Praze dne 6.8.2009 \hfill{}Jiří Kunčar

\tableofcontents{}

\newpage{}

\noindent Název práce: Informační systém pro jazykovou agenturu\\
Autor: Jiří Kunčar\\
Katedra (ústav): Ústav formální a aplikované lingvistiky\\
Vedoucí bakalářské práce: RNDr. Miroslav Spousta\\
e-mail vedoucího: Miroslav.Spousta@mff.cuni.cz\\


\noindent Abstrakt: Cílem práce je navrhnout a implementovat modulární
informační systém pro firmu zabývající se výukou jazyků.

\noindent Součástí informačního systému bude zejména:

\noindent - modul pro nabídku a prodej kurzů pro veřejnost, individuální
výuku, jazykovou výuku pro podniky, překlady a tlumočení 

\noindent - modul produkty (výuka, překlady, tlumočení), výuka (individuální
výuka, kurzy pro veřejnost, jazyková výuka pro podniky) 

\noindent - modul pro News (hromadné rozesílání mailem), rozvrh výuky,
třídní knihy, studijní materiály 

\noindent - modul dodavatelé (smlouvy s lektory, překladateli, tlumočníky,
měsíční výkazy odpracovaných hodin, měsíční přehledy nákladů na lektory,
překladatele, tlumočníky) 

\noindent - modul odběratelé (zápisy do kurzů pro veřejnost, objednávky,
smlouvy, přílohy faktur za období od-do) 

\noindent - modul pro testování znalostí (jednoduché testovací prostředí) 

\noindent - modul pro správu IS (uživatelé, jejich práva, přehled
změn).\\


\noindent Klíčová slova: informační systém, PHP, MySQL 

\vspace{10mm}


\noindent Title: Information system of a language school\\
Author: Jiří Kunčar\\
Department: Institute of Formal and Applied Linguistics\\
Supervisor: RNDr. Miroslav Spousta

\noindent Supervisor's e-mail address: Miroslav.Spousta@mff.cuni.cz\\


\noindent Abstract: The goal of the thesis is to design and implement
a modular information system for a company involved in teaching of
foreign languages.

\noindent The main parts of the information system will include, particularly: 

\noindent - a module for offer and sale of courses to the public and
individuals, language training for businesses, translation and interpretation 

\noindent - a module of products (teaching, translation and interpretation),
training (individual training courses for the public, language classes
for businesses) 

\noindent - a module for News (sending bulk mail), the schedule of
teaching, class books, study material 

\noindent - a module of suppliers (contracts with teachers, translators,
interpreters, monthly class-sheets, monthly reports on the cost of
staff translators and interpreters) 

\noindent - a module of customers (registration in courses, orders,
contracts, supplements to invoices for \textquotedbl{}from-to\textquotedbl{}
periods) 

\noindent - a module for testing of knowledge levels (a simple testing
environment) 

\noindent - a module for managing the IS (users, their rights and
a summary of changes)\\


\noindent Keywords: information system, PHP, MySQL

\newpage{}


\chapter*{Úvod}

Informační systémy hrají v rychle se rozvíjejícím prostředí svou nepostradatelou
roli a Internet jim poskytl výbornou platformu úmožňující další rozvoj.
S využíváním Internetu, ale zároveň rostou nároky na bezpečnost a
dostupnost uložených informací, které jsou pro chod firem životně
důležité.\\
Z původních jednoduchých aplikací na 


\section*{Definice a upřesnění pojmů}


\subsection*{Infomační systém}

Informační systém (IS) je systém pro sběr, udržování, zpracování a
poskytování informací a dat\cite{wikipedia}.


\section*{Cíle práce}

Cílem práce je zhodnocení procesu fungování firmy bez a s informačního
systému. Návrh a implementace informačního systému za použití vhodných
frameworků usnadňujících vývoj v použitých programovacích jazycích.


\section*{Obsah práce}
\begin{itemize}
\item Analýza úlohy
\item Požadavky
\item Existujicí implementace - IS na míru vs. hotová řešení
\item Technologie a frameworky

\begin{itemize}
\item http://en.wikipedia.org/wiki/Model-view-controller
\item PHP - CakePHP, jsMin, cssMini?
\item JS - Prototype, Livepipe, Script.aculo.us
\end{itemize}
\item Návrh vlastní implementace

\begin{itemize}
\item bezpečnost: Sanitize, 

\begin{itemize}
\item SQL Injection, Cross Site Scripting
\end{itemize}
\item výkonnost (použití cache)
\item optimalizace GET/POST požadavků na stránku
\end{itemize}
\item Programátorská dokumentace
\item Uživatelská dokumentace
\end{itemize}
Ve druhé kapitole této práce je provedena analýza úlohy s ohledem
na několik motivačních praktických příkladů. V této kapitole jsou
mimo jiné stanoveny požadavky na řešení a je uveden přehled obdobných
existujících implementací.

Ve třetí kapitole jsou stručně popsány technologie dále používané
v této práci a je ukázáno jejich použití.

Čtvrtá kapitola popisuje návrh vlastní implementace s ohledem na požadavky
stanovené v kapitole druhé. Jsou zde základní návrhová rozhodnutí
a důvody pro volbu konkrétních řešení.

Pátá kapitola obsahuje programátorskou dokumentaci. Je zde popsáno
technické řešení implementace a jsou nastíněny některé problémy, které
bylo při implementaci nutné řešit.

Uživatelská dokumentace je obsažena v šesté kapitole. Popisuje uživatelské
rozhraní vytvořených aplikací a ukazuje jejich použití na konkrétních
příkladech.

V závěru jsou přehledně shrnuty výsledky, kterých bylo při vývoji
systému dosaženo, a jsou naznačeny možnosti dalšího rozšíření.


\chapter{Název druhé kapitoly}


\section{Název první podkapitoly v druhé kapitole}


\section{Název druhé podkapitoly v druhé kapitole}


\chapter{Uživatelská dokumentace}

Role v systému: administrator, editor, odběratel, dodavatel (zaměstnanec,
brigádník či jiný subjekt vykonávající zadanou práci) a student (účastník
kurzu).


\section{Instalace}

Pro běh serverové aplikace je nutné mít sprovozněný program, který
úmožnuje zpracování zdrojových kódů a prezentaci výstupu protokolem
HTTP popřípadě HTTPS. Nejznámějším volně dostupným programem je Apache
{[}http://httpd.apache.org/{]}, který úmožnuje pomocí modulů {[}http://httpd.apache.org/modules/{]}
přidat podporu pro PHP.


\subsection{Stažení a instalace Apache}

Ze stránek projektu {[}http://httpd.apache.org/download.cgi{]} vyberte
odkaz vedoucí na požadovaný balík zdrojových kódů nebo předkompilovanou
aplikaci pro Váš operační systém. Pokud používáte některou z~moderních
linuxových distribucí, zkuste nejdříve projít repozitáře {[}@todo
vysvětlit{]}, zda se zde nenachází již hotový balík upravený pro snadnější
instalaci a~konfiguraci.

%
\begin{comment}
Debian a jemu podobní (Ubuntu, Kubuntu, ...):

apt-get install apache2

Gentoo

emerge apache2
\end{comment}
{}


\subsection{Konfigurace Apache}

Pro správnou funkci aplikace je potřeba doinstalovat, popřípadě pouze
povolit následující moduly: \textit{mod\_php5}, \textit{mod\_rewrite}
a \textit{mod\_ssl.} Pro správnou funkci zabezpečeného připojení {[}@todo
definovat zabezpečené připojení{]} je nutné vygenerovat certifikáty
a upravit konfiguraci stránek.


\subsection{Stažení a instalace MySQL serveru}




\section{Role v systému}

Má standardně nastavena veškerá přístupová práva ke všem modulům systému. 
\begin{thebibliography}{1}
\bibitem[1]{wikipedia}Wikipedia: \emph{Informační systém}, \\
http://cs.wikipedia.org/wiki/Informační\_systém
\end{thebibliography}

\end{document}
