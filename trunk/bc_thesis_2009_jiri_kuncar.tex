%% LyX 1.6.2 created this file.  For more info, see http://www.lyx.org/.
%% Do not edit unless you really know what you are doing.
\documentclass[12pt,czech,notitlepage]{report}
\usepackage[T1]{fontenc}
\usepackage[utf8]{inputenc}
\pagestyle{plain}
\setcounter{secnumdepth}{1}
\setcounter{tocdepth}{1}
\usepackage{babel}

\usepackage{url}
\usepackage{graphicx}
\usepackage[unicode=true, pdfusetitle,
 bookmarks=true,bookmarksnumbered=false,bookmarksopen=false,
 breaklinks=false,pdfborder={0 0 1},backref=false,colorlinks=false]
 {hyperref}


%%%%%%%%%%%%%%%%%%%%%%%%%%%%%% LyX specific LaTeX commands.
\newcommand{\noun}[1]{\textsc{#1}}

%%%%%%%%%%%%%%%%%%%%%%%%%%%%%% User specified LaTeX commands.
\usepackage{a4wide}
\usepackage[left=4cm]{geometry}

\begin{document}
\begin{titlepage}\vspace{15mm}


\begin{center}
{\large Univerzita Karlova v Praze}\\
{\large Matematicko-fyzikální fakulta\vspace{5mm}
}
\par\end{center}{\large \par}

\begin{center}
\textbf{\Large BAKALÁŘSKÁ PRÁCE\vspace{10mm}
}
\par\end{center}{\Large \par}

\begin{center}
\includegraphics[scale=0.3]{logo}\vspace{15mm}

\par\end{center}

\begin{center}
{\Large Jiří Kunčar \vspace{5mm}
}
\par\end{center}{\Large \par}

\begin{center}
\textbf{\Large Informační systém pro jazykovou agenturu\vspace{5mm}
}
\par\end{center}{\Large \par}

\begin{center}
\textbf{\large Ústav formální a aplikované lingvistiky\vspace{15mm}
}
\par\end{center}{\large \par}

\noindent \begin{center}
{\large Vedoucí bakalářské práce: RNDr. Miroslav Spousta \vspace{1mm}
}
\par\end{center}{\large \par}

\noindent \begin{center}
{\large Studijní program: informatika, správa počítačových systémů\vspace{20mm}
}
\par\end{center}{\large \par}

\begin{center}
{\large 2009}
\par\end{center}{\large \par}

\end{titlepage}\setcounter{page}{2}

\vspace{10mm}


\noindent Děkuji panu RNDr. Miroslavu Spoustovi za pomoc, připomínky,
cenné rady a za odborné vedení bakalářské práce. Dále bych rád poděkoval
firmě Primalingua s.r.o. za poskytnutí prostředků na vývoj aplikace.
Speciální poděkování patří především paní majitelce Mgr. Miluši Psotové
a paní RNDr. Jitce Kunčarové, která vývoj informačního systému iniciovala.

\vfill{}


\noindent Prohlašuji, že jsem svou bakalářskou práci napsal samostatně
a výhradně s použitím citovaných pramenů. Souhlasím se zapůjčováním
práce a jejím zveřejňováním.\bigskip{}


\noindent V Praze dne 6.8.2009 \hfill{}Jiří Kunčar

\tableofcontents{}

\newpage{}

\noindent Název práce: Informační systém pro jazykovou agenturu\\
Autor: Jiří Kunčar\\
Katedra (ústav): Ústav formální a aplikované lingvistiky\\
Vedoucí bakalářské práce: RNDr. Miroslav Spousta\\
e-mail vedoucího: Miroslav.Spousta@mff.cuni.cz\\


\noindent Abstrakt: Cílem práce je navrhnout a implementovat modulární
informační systém pro firmu zabývající se výukou jazyků.

\noindent Součástí informačního systému bude zejména:

\noindent - modul pro nabídku a prodej kurzů pro veřejnost, individuální
výuku, jazykovou výuku pro podniky, překlady a tlumočení 

\noindent - modul produkty (výuka, překlady, tlumočení), výuka (individuální
výuka, kurzy pro veřejnost, jazyková výuka pro podniky) 

\noindent - modul pro News (hromadné rozesílání mailem), rozvrh výuky,
třídní knihy, studijní materiály 

\noindent - modul dodavatelé (smlouvy s lektory, překladateli, tlumočníky,
měsíční výkazy odpracovaných hodin, měsíční přehledy nákladů na lektory,
překladatele, tlumočníky) 

\noindent - modul odběratelé (zápisy do kurzů pro veřejnost, objednávky,
smlouvy, přílohy faktur za období od-do) 

\noindent - modul pro testování znalostí (jednoduché testovací prostředí) 

\noindent - modul pro správu IS (uživatelé, jejich práva, přehled
změn).\\


\noindent Klíčová slova: informační systém, PHP, MySQL 

\vspace{10mm}


\noindent Title: Information system of a language school\\
Author: Jiří Kunčar\\
Department: Institute of Formal and Applied Linguistics\\
Supervisor: RNDr. Miroslav Spousta

\noindent Supervisor's e-mail address: Miroslav.Spousta@mff.cuni.cz\\


\noindent Abstract: The goal of the thesis is to design and implement
a modular information system for a company involved in teaching of
foreign languages.

\noindent The main parts of the information system will include, particularly: 

\noindent - a module for offer and sale of courses to the public and
individuals, language training for businesses, translation and interpretation 

\noindent - a module of products (teaching, translation and interpretation),
training (individual training courses for the public, language classes
for businesses) 

\noindent - a module for News (sending bulk mail), the schedule of
teaching, class books, study material 

\noindent - a module of suppliers (contracts with teachers, translators,
interpreters, monthly class-sheets, monthly reports on the cost of
staff translators and interpreters) 

\noindent - a module of customers (registration in courses, orders,
contracts, supplements to invoices for \textquotedbl{}from-to\textquotedbl{}
periods) 

\noindent - a module for testing of knowledge levels (a simple testing
environment) 

\noindent - a module for managing the IS (users, their rights and
a summary of changes)\\


\noindent Keywords: information system, PHP, MySQL

\newpage{}


\chapter*{Úvod}

Informační systémy hrají v rychle se rozvíjejícím prostředí svou nepostradatelou
roli a Internet jim poskytl výbornou platformu úmožňující další rozvoj
v dříve nepříliš využívaném prostředí. Úmožňují snadné a rychlé zpracování,
vyhledávání a publikovaní informací, které se díky němu stávají dostupné
pro širší spektrum zákazníků, dodavatelů i samotných zaměstanců. 

S využíváním Internetu, ale zároveň rostou nároky na funkce, bezpečnost
a dostupnost uložených informací, které jsou pro chod firem životně
důležité. Toto zvyšuje požadavky na robustnost řešení včetně nároků
na nástroje použité při implementaci systému.

Před zavádění informačního systému je nutné stanovit, čeho chce firma
s využitím nového systému v daném čase dosáhnout. Tomu by měla pomoci
důkladná analýza firemních procesů a jejich optimalizace. 

\begin{center}
\textit{,,Kdo nepozná vlastní firemní procesy, nemůže je zlepšovat.''}\textit{\emph{{[}citát{]}}}
\par\end{center}

Z původních jednoduchých aplikací na 

Informačný systém, Client/Server. 


\section*{Cíle práce}

Cílem práce je návrh a implementace informačního systému usnadňující
činnost a spolupráci více subjektů za použití vhodných opensource
technologií a frameworků v použitých programovacích jazycích.

Cílem práce není vytvořit, vzhledem k náročným právním úpravám a častým
změnám, komplexní účetní program, ale systém úmožní vytvořit přehledy
pro účetní evidenci, které mohou být zavedeny do některého již existujícího
programu.


\section*{Obsah práce}
\begin{itemize}
\item Analýza úlohy
\item Požadavky
\item Existujicí implementace - IS na míru vs. hotová řešení
\item Návrh vlastní implementace

\begin{itemize}
\item bezpečnost: Sanitize, 

\begin{itemize}
\item SQL Injection, Cross Site Scripting
\end{itemize}
\item výkonnost (použití cache)
\item optimalizace GET/POST požadavků na stránku
\end{itemize}
\item Programátorská dokumentace
\item Uživatelská dokumentace
\end{itemize}
Poznamky k prepsani....

Ve druhé kapitole této práce je provedena analýza úlohy s ohledem
na několik motivačních praktických příkladů. V této kapitole jsou
mimo jiné stanoveny požadavky na řešení a je uveden přehled obdobných
existujících implementací.

Ve třetí kapitole jsou stručně popsány technologie dále používané
v této práci a je ukázáno jejich použití.

Čtvrtá kapitola popisuje návrh vlastní implementace s ohledem na požadavky
stanovené v kapitole druhé. Jsou zde základní návrhová rozhodnutí
a důvody pro volbu konkrétních řešení.

Pátá kapitola obsahuje programátorskou dokumentaci. Je zde popsáno
technické řešení implementace a jsou nastíněny některé problémy, které
bylo při implementaci nutné řešit.

Uživatelská dokumentace je obsažena v šesté kapitole. Popisuje uživatelské
rozhraní vytvořených aplikací a ukazuje jejich použití na konkrétních
příkladech.

V závěru jsou přehledně shrnuty výsledky, kterých bylo při vývoji
systému dosaženo, a jsou naznačeny možnosti dalšího rozšíření.


\chapter{Analýza úlohy}


\section{Požadavky klienta\label{sec:Po=00017Eadavky-klienta}}

Na začátku stála žádost manažerky agentury zjednodušit proces zpracování
měsíčních výkazů lektorů, překladatelů a tlumočníků tvořících podklady
pro mzdy a fakturaci služeb. Tyto měsíční výkazy neměly jednotný vzor
a ani nebylo jednoduše možné, bez znalostí místních poměrů přiřadit
vykázanou činnost k jednotlivým produktům.


\subsection{Zavedení jednotných identifikátorů}

Kvůli výše uvedným problémům se začalo s postupným zaváděním jednotných
identifikátorů závazných pro všechny zúčastněné strany. Tento krok
byl ze začátku velmi těžce snášen a trvalo několik měsíců, než se
tento proces tvorby ustálil a začal být všemi akceptován.


\subsection{Měsíční výkazy}

Vzhledem k velkým rozdílům mezi jednotlivými lektory, překladateli
a tlumočníky, bylo rozhodnuto, že se v první fázi začne s jednotým
papírovým formulářem {[}@todo odkaz na prilohu{]}. Ten bude následně
zkontrolován administrativním pracovníkem a vložen do systému, aby
se předešlo chybám.


\subsection{Podklady pro fakturaci}

Se vzrůstajícím objemem překladů a odučených hodin přestávalo být
únosné ruční vytváření měsíčních přehledů pro zákazníky. Ti si přáli
být informování nejen o počtu odučených hodit, ale i o všech změnách
zavedeném měsíčním rozvrhu a případném suplování. Zárověn musely být
v systému zachyceny vazby mezi produktem, zákazníkem a vlastními účastníky
tak, aby bylo možné stanovit výslednou cenu zakázky, která může být
závislá nejen od počtu odučených hodin či přeložených stran, ale i
od počtu účastníků.


\subsection{Oddělení reálných dat více agentur}

Vzhledem ke složitým poměrům v agentuře bylo potřeba oddělit evidenci
zakázek vyřizovaných manažerem pro různé agentury a fyzické osoby,
jenž s agenturami úzce spolupracují. 


\subsection{Úzká provázanost s webem}

Evidence hodin, rozvrhy učeben


\subsection{Vícejazykový systém}

Vzhledem k tomu, že je systém určen primárně pro jazykové agentury,
kde se počítá s komunikací se zákazníky v jiném než českém jazyce,
bylo k tomu přizpůsobit systém již od počátku.


\section{Definice a upřesnění pojmů}


\subsection{Agentura}


\subsection*{Infomační systém}

Informační systém (IS) je systém pro sběr, udržování, zpracování a
poskytování informací a dat\cite{wikipedia}.


\subsection{Produkt}


\subsection{Zakázka}

CSS


\chapter{Existující implementace}

Existujicí aplikace lze rozdělit na několik skupin, kde ovšem žádná
nepokrývá veškeré požadavky v plném rozsahu. První se více zaměřuje
na správu webového obsahu. Označované anglickou zkratkou CMS\label{CMS}
- Content Management System, před kterou se někdy přídává W označující
webové systémy. Tyto systémy lze rozdělit do podskupin podle způsobu,
jakým prezentují uložená data.
\begin{description}
\item [{Offline~zpracování}] Server vygeneruje statické HTML před samotnou
publikací, proto takové systémy nepořebují, aby server aplikoval šablony
na data při každém požadavku. Výhody jsou zjevné pro vytížené CMS
používané převážně pro čtení dat. Naopak při častých změnách dat můžou
být tyto systémy pomalé nebo neaktuální, proto je nutné si jejich
nasazení řádně rozmyslet. Příkladem takového systému je například
Vignette CMS%
\footnote{\url{http://www.vignette.com/}%
}.
\item [{Online~zpracování}] Server generuje prezentační data až na základě
požadavků klienta. Pro keše
\item [{E-learning}] ... Moodle
\item [{ciele:}]~
\item [{{*}}] strategické (plánovanie investícií…) {*} taktické (vedenie,
kontrola rozpočtu…) {*} operatívne (každodenná rutina)
\item [{Dôležité}] sú tiež úlohy IS:
\item [{{*}}] manažérske (EIS - Executive IS) {*} taktické (DSS - Decision
Support System) {*} vedenie (MIS - Management IS) {*} expertné (KWS
- Knowledge Work System) {*} kancelárske (OIS - Office IS) {*} operatívne
o TPS - transakčné (banky, ...) o CRM - vzťahy so zákazníkmi o RIS
- rezervačné systémy o CAM - konštrukčné (CAD, ...) o GIS - geografické
systémy
\end{description}

\chapter{Návrh řešení}

Evidence produktů

Podklady pro fakturaci dodavatelům a mzdy

Finanční vyrovnání subjektů v IS


\chapter{Použité technologie a frameworky}

Použité technologie byly částečně determinovány požadavy klienta\ref{sec:Po=00017Eadavky-klienta}.
Tím, že se mělo jednat o systém využívající výhradně open-souce technologie
tak, aby nebyly zvyšovány náklady na nákup licencí.


\section{Server}

Tento webový informační systém využívá technologi PHP (jazyk, interpret
a knihovny), která vychází se skriptovacího víceúčelového jazyka,
jenž byl původně vyvinut pro tvorbu dynamických webových stránek.
Z tohoto využití vznikla i zkatka z anglických slov \emph{Personal
Home Page}, které byly nahrazeny slovy \emph{PHP: Hypertext Preprocessor}
dající vznik rekurzivní zkratce%
\footnote{http://cs.wikipedia.org/wiki/Rekurzivní\_zkratka%
}.

Výhodou použití PHP je existence interpretu pro různé operační systémy
a podobnost jeho syntaxe s C, Javou.

Nevýhodou, která brzdí dalšímu rozvoji a rozšíření, je absence normy
(k datu vydání BP). Jazyk je tak de facto standardizovaný interpretem
a množstvím lidí%
\footnote{dle statistiky na http://www.php.net/usage.php%
}, kteří jej využívají. I když existují mnohé polemiky a živé diskuze
mezi jeho zastánci a odpůrci o jeho výkonnosti, bezpečnosti a vhodnosti
pro velké projekty, existují vyjímky%
\footnote{Facebook, YouTube, Wikipedia (MediaWiki) a další%
}, které tyto názory vyvrací a zároveň se podílejí na vývoji, a tak
se snaží přispět k jeho větší výkonnosti a bezpečnosti.

Vývoj jazyka sebou nese i stinné stránky. Mezi ty nejpodstatnější
patří úpravy API%
\footnote{anglická zkratka \emph{Application Programming Interface}, označuje
sadu funkcí, procedur či tříd programu či knihovny, jenž mohou být
využívány programátorem %
} některých vestavěných funkcí a změna syntaxe. To může zapříčinit,
že po aktualizaci interpretu jazyka, přestanou fungovat některé části
nebo celá aplikace. Řešením ovšem není zůstávat na několik let staré
verzi, ve které mohly být objeveny chyby. 

Pokud programátor nechce nebo nemůže přepisovat kód může využít služeb
něktrého z frameworků nad daným jazykem, který se snaží tyto rozdíly
ve verzích zakrýt. Další nespornou výhodou používání frameworků spočívá
jednodušším vývoji aplikací a minimalizací rizika chyb v jinak ručně
psaném jádru aplikace. Toto je zajištěno pouze pokud má kvalitní a
úplnou dokumentaci a je zastřešen silnou komunitou nebo společností
zajištující jeho vývoj.

Při výběru byl proto důraz kladen hlavně na kvalitní dokumentaci,
rozšířitelnost a možnost práce s různými relačními databázemi. Frameworky,
které splňují vetšinu požadavků jsou Zend%
\footnote{\href{http://framework.zend.com/}{http://framework.zend.com/}%
}, Symfony a CakePHP.


\subsection{CakePHP}

Nám dává příležitost věnovat se návrhu schématu. MVC - Model-View-Controller
- http://en.wikipedia.org/wiki/Model-view-controller

Odstínění od způsobu práce odlišnými zdroji dat. mezi relačními databázemi
(MySQL, PostgreSQL, Oracle, MSSQL, ...) a dokonce i definování vlastních
zdrojů dat ať už z lokálních zdrojů s přímým přístupem (formátované
soubory např. csv, xml) nebo za pomocí API k on-line službám (LDAP,
twitter, IMAP).

Díky množství napsaného a otestovaného kódu, není potřeba psát již
jednou napsané části, ale možné šetřit lidské zdroje na vývoj. Tím
není myšleno prosté skládání kusů posbíraného kódu, ale smysluplného
využívání dostupných knihoven a pluginů do programovacího jazyka nebo
frameworku. 

Díky těmto úsporám je možné se zaměřit na ergonomii dané aplikace
a její možnou optimalizaci, která je ovšem limitována výkonností použitého
skriptovacího jazyka. Tato omezení lze, minimalizovat udržováním částí
zpracovaného zdrojového kódu v paměti a kešováním%
\footnote{z anglického slova \emph{cache}, označuje vyrovnávací paměť%
} nebo kompresí výstupu. 

Tuto nepříjemnou vlastnost skriptovacích jazyků lze řešit předkompilací
zdrojových kódů, nebo vhodným využíváním ,,keší''.

APC, Xcache, File

Další možnou optimalizací, která sice přímo nesouvisí s CakePHP, ale
je v něm snadno implementovatelná, je snížení počtu požadavků na stránku.
S používáním javascriptových frameworků a knihoven se snadno může
stát, že počet vkládaných odkazů na skripty a kaskádové styly (CSS)
se vyšplhá až k desítkám a začne se neúměrně prodlužovat doba potřebná
na stažení všech potřebných částí. Zvyšuje se tak počet požadavků
na server a díky režii protokolu HTTP je ve výsledku stažen větší
objem dat. Tento problém úspěšně řeší projekty \emph{jsMin}%
\footnote{\href{http://code.google.com/p/jsmin-php/}{http://code.google.com/p/jsmin-php/}%
} a \emph{CSSTidy}%
\footnote{\href{http://csstidy.sourceforge.net/}{http://csstidy.sourceforge.net/}%
}.


\subsection{MySQL}


\section{Klient}

Pro vlastní běh aplikační logiky je možné se spolehnout, že serverová
část aplikace bude zpracována jednou verzí PHP interpretu v uzavřeném
a otestovaném prostředí. Naproti tomu klientská část bude prezentována
na odlišných operačních systémech v mnoha prohlížečích nejrůznějších
verzích. Základními požadavky kladené na prohlížeč jsou:
\begin{itemize}
\item XHTML 1.0%
\footnote{\href{http://www.w3.org/TR/xhtml1/}{http://www.w3.org/TR/xhtml1/}%
}
\item CSS 2.1%
\footnote{\href{http://www.w3.org/TR/CSS2/}{http://www.w3.org/TR/CSS2/}%
}
\item JavaScript%
\footnote{JavaScript je dialektem ECMASriptu. Minimálním implementovaným standardem
by měl být ECMA-262, revize 3:\href{http://www.ecma-international.org/publications/standards/Ecma-262.htm}{http://www.ecma-international.org/publications/standards/Ecma-262.htm}.%
}
\end{itemize}
S XHTML a CSS prohlížeče problémy nemívají. Horší je to, ale s implementacemi
Javascriptu. Toto bylo vyřešeno díky provázanosti CakePHP a javascriptového
frameworku Prototype{[}@todo odkaz{]} a jeho rozšíření Script.aculo.us{[}@todo
odkaz{]}, který se snaží zakrýt rozdíly mezi prohlížeči.

Prototype není jediným či nejlepším frameworkem. Existují i jiné,
které mají rozsáhlejší schopnosti v oblasti používaní dotazovacího
jazyka XPath nebo tvorby GUI%
\footnote{z anglických slov \emph{Graphical User Interface}, uživatelské prostředí,
jenž úmožnuje uživateli ovládat aplikaci pomocí grafických ovládacích
prvků%
} 


\subsection{Prototype a jeho rozšíření}

Prototype úmožnuje 
\begin{itemize}
\item JS - Prototype, Script.aculo.us, Livepipe
\end{itemize}

\chapter{Programátorská dokumentace}

V následující kapitole je popsáno a vysvětleno databázové schéma,
základní adresářová struktura projektu, a metody použitých tříd.


\section{Databázové schéma}

Normalizace???


\section{Adresářová struktura}


\chapter{Uživatelská dokumentace}

Tato uživatelská dokumentace si klade za cíl stručně přiblížit čtenáři
výhody informačního systému a jeho záklaními způsoby ovládání. Jednotlivých
kapitolách je vysvětleno, jak má správce postupovat od úvodní instalace,
přes inicializaci databáze, uložení informací o uživatelích, přidání
produktů až po tisk účetních podkladů. 

Dodavatelům a zaměstnancům je názorně předvedeno, jak správně a včas
vyplnit měsíční výkazy a jak vést evidenci docházky účastníků kurzů. 

Zákazníkům je vysvětleno, jak jednoduše zjistit, kolik z objednaných
služeb již bylo zaplaceno a zkontrolovat průběh aktuálních kurzů či
počet přeložených stran překladu.

Studenti


\section{Instalace serveru}

Pro běh serverové aplikace je nutné mít sprovozněný program, který
úmožnuje zpracování zdrojových kódů a prezentaci výstupu protokolem
HTTP popřípadě HTTPS. Nejznámějším volně dostupným programem je Apache
{[}http://httpd.apache.org/{]}, který úmožnuje pomocí modulů {[}http://httpd.apache.org/modules/{]}
přidat podporu pro jazyk PHP nutný k běhu IS.


\subsection{Stažení a instalace Apache}

Pokud používáte některou z~moderních linuxových distribucí, zkuste
nejdříve projít repozitáře {[}@todo vysvětlit{]}, zda se zde nenachází
již hotový balík upravený pro snadnější instalaci a~konfiguraci.

%
\begin{minipage}[t]{1\columnwidth}%
Debian a jemu podobní (Ubuntu, Kubuntu, ...):

apt-get install apache2

Gentoo

emerge apache2%
\end{minipage}

Pokud jste požadovaný balík nenašli či používáte jiný operační systém,
můžete ze stránek projektu%
\footnote{\href{http://httpd.apache.org/download.cgi}{http://httpd.apache.org/download.cgi}%
} vybrat odkaz vedoucí na požadovaný balík zdrojových kódů nebo předkompilovanou
aplikaci pro Váš operační systém.


\subsection{Konfigurace Apache}

Pro správnou funkci aplikace je potřeba doinstalovat, popřípadě pouze
povolit následující moduly: \textit{mod\_php5}, \textit{mod\_rewrite}
a \textit{mod\_ssl.} Pro správnou funkci zabezpečeného připojení {[}@todo
definovat zabezpečené připojení{]} je nutné vygenerovat certifikáty
a upravit konfiguraci stránek.


\subsection{Stažení a instalace MySQL serveru\label{sub:Sta=00017Een=0000ED-a-instalace}}

uživatelské jméno: isadb\\
heslo: isapass


\section{Umístění aplikace}

Pokud máme server správně nakonfigurovaný, zkopírujeme složku s aplikací
do adresáře určeného konfigurací Apache (obvykle /var/www či C:\textbackslash{}\textbackslash{}Program
Files\textbackslash{}apache2\textbackslash{}www). Dále je potřeba
nastavit přihlašovací údaje k databázi podle \ref{sub:Sta=00017Een=0000ED-a-instalace}předchozí
části.


\section{Inicializace databáze}

V souboru \%CDROM\%/app/config/sql/isa\_init.sql se nachází MySQL
5.0+ kompatibilní skript, který vytvoří tabulky a naplní je daty nutnými
k prvnímu přihlášení administrátora.


\section{Upřesňující informace}

Pro další čtení manuálu je potřeba upřesnit několik důležitých pojmů,
které se budou dále vyskytovat.
\begin{description}
\item [{Systémová~skupina}] je nutná pro správné fungování IS.
\end{description}
Tato sekce je rozdělena podle rolí definovaných v IS.
\begin{description}
\item [{Správce:}] pověřený uživatel s plnými právy ke všem modulům systému.
\item [{Editor:}] osoba s omezenými právy k editaci vybraných modulů.
\item [{Dodavatelé:}] zaměstnanec, brigádník či jiný subjekt vykonávající
zadanou práci.
\item [{Odběratelé:}] @todo
\item [{Poskytovatelé:}] subjekty, které jsou vedeny ,,pod jednou střechou''
a sdílí část informací.
\item [{Účastníci:}] studenti jednotlivých kurzu. 
\end{description}
Výše popsané role mohou být změněny či zakázány administrátorem systému.
Čtenáři je doporučeno číst pouze části, jenž se ho týkají.


\section{Správce (hlavní manažer)}

Má standardně nastavena veškerá přístupová práva ke všem modulům systému. 


\subsection{Správa IS}

Tento modul obsahuje moduly pro nastavení jednotlivých částí systému,
správu uživatelů, skupin, práv a ostatních číselníků (daně, místnosti,
kategorie produktů).


\subsubsection{Uživatelé (Users)\label{sub:U=00017Eivatel=0000E9-(Users)}}

Modul \noun{Uživatelé} nabízí veškeré nastavení potřebné pro definování
možností při užívání systému osobami majícími vztah k IS. Dále jsou
zde uchovány veškeré osobní informace důvěrného charakteru.
\begin{description}
\item [{Seznam~uživatelů~{[}/admin/Users/index{]}:}] V horní části se
nachází odkazy na akce související s uživately. Filtry pro práci se
seznamem uživatelů jsou umístěny nad hlavní tabulkou a v hlavičce
tabulky s funkcí ,,našeptávače''.
\item [{Přidání~uživatele~{[}/admin/Users/add{]}:}] Jedinnou povinnou
položkou je \emph{zobrazovanné jméno}, která slouží jako popisek ve
všech výběrových filtrech. Ostatní položky můžete vyplnit až při jejich
potřebě ve výpisu. Volba \emph{Aktivní} slouží k aktivaci uživatelského
účtu. Uživatel je schopen se přihlásit pouze pokud je jeho účet aktivní.
\item [{Editace~uživatele:}] Vyberte odpovídající skupinu k editaci a
pro uložení změn použijte tlačítko \emph{Uložit}. V případě chybového
hlášení zkontrolujte všechny skupiny údajů.
\item [{Zobrazení~uživatele:}] @todo
\item [{Nastavení~či~změna~hesla:}] V přehledu uživatele klikněte na
odkaz Změnit heslo a vyplňte nové heslo do obou kolonek. Správce může
editovat hesla všem uživatelům a měl by je o této změně informovat
zabezpečeným kanálem{[}@todo definovat zabezpeceny kanal{]}, aby se
předešlo zneužití jejich účtu.
\end{description}

\subsubsection{Skupiny a oprávnění}

V tomto modulu můžete přidávat a mazat uživatelské skupiny a oprávnění.
Pro zachování správného fungování systému nemažte tyto systémové skupiny:
customers, employees, providers, students a admin.

Při implicitní nastavení má každá ze systémových skupin práva definované
a přiřazené oprávnění k prefixovaným akcím jednotlivých modulů. Tyto
oprávnění jsou definovány tímto způsobem: \emph{{*}:customers\_{*},
{*}:employees\_{*}, {*}:providers\_{*}, {*}:students\_{*}, {*}:admin\_{*}}
a říkají, že daná skupina může v libovolném modulu spouštět akce začínajícím
jejich jménem. V příkladu byl použit expanzní znak ,,{*}'', který
je možné používat k nahrazení libovoného počtu libovolných znaků,
a ,,:'', jenž určuje hranici mezi modulem a akcí (např. oprávnění
\emph{Users:{*}\_view} úmožní zobrazit všechny prefixované akce \emph{view}
v modulu \noun{Uživatelé}).

Každá skupina může může mít přiřazeno více definovaných oprávnění
a je na správci, jak s nimi bude zacházet.


\subsubsection{Číselníky}

V této části se seznámíme se všemi číselníky, jejich funkcemi v systému
a možností editace. Všechy dále uvedené moduly obsahují automaticky
generovaný číselný identifikátor (dále jen \emph{id}), který slouží
pro interní potřeby systému a není možné ho změnit. Dále je potřeba
upozornit na fakt, že vytvořené položky v číselníku, které již byly
v systému přizazeny nějakým záznamům, nelze vymazat. Vymazání je umožněno
až když je číselník u daných záznamů změněn.
\begin{description}
\item [{Kategorie\label{kategorie}}] obsahuje název, zkratku, jednoduchý
slovní popis a zašktávací políčko určující, zda se produkty v této
kategorii považují za veřejné a má se zobrazovat jejich rozvh na webových
stránkách. Účel číselníku spočíva v rozčlenění množství produktů do
skupin, podle kterých lze vytvářet tiskové sestavy (přílohy faktur,
rozvrhy, měsíční přehledy nákladů, ...).
\item [{Místa\label{m=0000EDsta}}] obsahují název a adresu. Slouží k odkazu
na místo v událostech.
\item [{Štítky\label{st=0000EDtky}}] obsahují název a odkaz na nadřazený
štítek, pro možnost tvorby hierarchické struktury článků.
\end{description}

\subsection{Produkty}

Modul \noun{Produkty} je závislý na správně rozdělených uživatelích
do systémových skupin. Rovněž je doporučeno předvyplnit číselník kategorií
produktů.


\subsubsection{Vytvoření nového produktu (jednoduchá verze)}

Vyberte kategorii a poskytovatele. Dále vyplňte název produktu a infomaci
o datu zahájení a předpokládaném či žádaném datu ukončení prací. Dále
si řekněme, jak se vypočítá výsledná cena (bez DPH) pro zákazníka.
Od toho se bude odvíjet další vyplňování formuláře.

Stanovte si, jak budete daný produkt nabízet a od čeho se odvíjí náklady.
V~případě, že se náklady odvíjí od počtu participujících osob zašrtněte
možnost \emph{Počítat účastníky}. Pokud nevíte dobředu kolik hodin
bude odučeno či kolik normostran bude účtováno a náklady na ně nejsou
fixní, zaškrtněte možnost \emph{Počítat množství}. Celková cena bude
spočítána takto:
\begin{itemize}
\item jednotková cena je vynásobena počtem účastníků pokud byla odpovídající
volba zaškrtnuta
\item mezisoučet je vynásoben součtem odpracovaných jednotek v událostech
ve výkazech přiřazených v produktu pokud byla volba \emph{počítat
množství} zaškrtnuta
\item @todo možná konverze
\end{itemize}
Teď už zbývá doplnit zbývající povinné položky a to jednotku a jednotkovou
cenu.


\subsection{Výkazy}


\subsubsection{Vytvoření~a~přiřazení~výkazu~k~produktu }

Vytvoření a přiřazení výkazu k produktu provádějte pouze pokud je
produkt, ke kterému chcete přidat výkaz, již vytvořen. 

V menu klikněte na položku \emph{Přidat výkaz} a vyplňte následný
přehledný formulář. U dodavatele se rozlišuje se i typ smlouvy uvedený
v závorce. Sazba a jednotka jsou rovněž povinné položky.

Jinou možností, jak přiřadit nový výkaz, je přes modul \noun{Produkty}
(nebo přes ,,rychlé hledání''), kde lze snadno vyhledat daný produkt.
Zobrazte si detaily nalezeného produktu a ve skupině \emph{Přiřazené
výkazy} klikněte na odkaz \emph{Přidat výkaz}.


\subsubsection{Přidání události k výkazu}


\subsection{Tipy na urychlení práce}

Zvláštní formulář na přidání produktu spolu se zákazníkem i dodavatelem
úmožní rychlejší zadávání většího množství nových produktů.


\section{Dodavatel}

zadávání výkazu,

evidence docházky do kurzů


\chapter*{Závěr}

Zavedení informačního systému je běh na dlouhou trať ... příprava
dodavatelů i odběratelů na změny ve způsobu vykazování práce ...
\begin{thebibliography}{IS}
\bibitem[1]{informační systémy}

\bibitem[IS]{wikipedia}Wikipedia: \emph{Informační systém}, \\
http://cs.wikipedia.org/wiki/Informační\_systém
\end{thebibliography}

\end{document}
